% Options for packages loaded elsewhere
\PassOptionsToPackage{unicode}{hyperref}
\PassOptionsToPackage{hyphens}{url}
\documentclass[
]{article}
\usepackage{xcolor}
\usepackage[margin=1in]{geometry}
\usepackage{amsmath,amssymb}
\setcounter{secnumdepth}{-\maxdimen} % remove section numbering
\usepackage{iftex}
\ifPDFTeX
  \usepackage[T1]{fontenc}
  \usepackage[utf8]{inputenc}
  \usepackage{textcomp} % provide euro and other symbols
\else % if luatex or xetex
  \usepackage{unicode-math} % this also loads fontspec
  \defaultfontfeatures{Scale=MatchLowercase}
  \defaultfontfeatures[\rmfamily]{Ligatures=TeX,Scale=1}
\fi
\usepackage{lmodern}
\ifPDFTeX\else
  % xetex/luatex font selection
\fi
% Use upquote if available, for straight quotes in verbatim environments
\IfFileExists{upquote.sty}{\usepackage{upquote}}{}
\IfFileExists{microtype.sty}{% use microtype if available
  \usepackage[]{microtype}
  \UseMicrotypeSet[protrusion]{basicmath} % disable protrusion for tt fonts
}{}
\makeatletter
\@ifundefined{KOMAClassName}{% if non-KOMA class
  \IfFileExists{parskip.sty}{%
    \usepackage{parskip}
  }{% else
    \setlength{\parindent}{0pt}
    \setlength{\parskip}{6pt plus 2pt minus 1pt}}
}{% if KOMA class
  \KOMAoptions{parskip=half}}
\makeatother
\usepackage{color}
\usepackage{fancyvrb}
\newcommand{\VerbBar}{|}
\newcommand{\VERB}{\Verb[commandchars=\\\{\}]}
\DefineVerbatimEnvironment{Highlighting}{Verbatim}{commandchars=\\\{\}}
% Add ',fontsize=\small' for more characters per line
\usepackage{framed}
\definecolor{shadecolor}{RGB}{248,248,248}
\newenvironment{Shaded}{\begin{snugshade}}{\end{snugshade}}
\newcommand{\AlertTok}[1]{\textcolor[rgb]{0.94,0.16,0.16}{#1}}
\newcommand{\AnnotationTok}[1]{\textcolor[rgb]{0.56,0.35,0.01}{\textbf{\textit{#1}}}}
\newcommand{\AttributeTok}[1]{\textcolor[rgb]{0.13,0.29,0.53}{#1}}
\newcommand{\BaseNTok}[1]{\textcolor[rgb]{0.00,0.00,0.81}{#1}}
\newcommand{\BuiltInTok}[1]{#1}
\newcommand{\CharTok}[1]{\textcolor[rgb]{0.31,0.60,0.02}{#1}}
\newcommand{\CommentTok}[1]{\textcolor[rgb]{0.56,0.35,0.01}{\textit{#1}}}
\newcommand{\CommentVarTok}[1]{\textcolor[rgb]{0.56,0.35,0.01}{\textbf{\textit{#1}}}}
\newcommand{\ConstantTok}[1]{\textcolor[rgb]{0.56,0.35,0.01}{#1}}
\newcommand{\ControlFlowTok}[1]{\textcolor[rgb]{0.13,0.29,0.53}{\textbf{#1}}}
\newcommand{\DataTypeTok}[1]{\textcolor[rgb]{0.13,0.29,0.53}{#1}}
\newcommand{\DecValTok}[1]{\textcolor[rgb]{0.00,0.00,0.81}{#1}}
\newcommand{\DocumentationTok}[1]{\textcolor[rgb]{0.56,0.35,0.01}{\textbf{\textit{#1}}}}
\newcommand{\ErrorTok}[1]{\textcolor[rgb]{0.64,0.00,0.00}{\textbf{#1}}}
\newcommand{\ExtensionTok}[1]{#1}
\newcommand{\FloatTok}[1]{\textcolor[rgb]{0.00,0.00,0.81}{#1}}
\newcommand{\FunctionTok}[1]{\textcolor[rgb]{0.13,0.29,0.53}{\textbf{#1}}}
\newcommand{\ImportTok}[1]{#1}
\newcommand{\InformationTok}[1]{\textcolor[rgb]{0.56,0.35,0.01}{\textbf{\textit{#1}}}}
\newcommand{\KeywordTok}[1]{\textcolor[rgb]{0.13,0.29,0.53}{\textbf{#1}}}
\newcommand{\NormalTok}[1]{#1}
\newcommand{\OperatorTok}[1]{\textcolor[rgb]{0.81,0.36,0.00}{\textbf{#1}}}
\newcommand{\OtherTok}[1]{\textcolor[rgb]{0.56,0.35,0.01}{#1}}
\newcommand{\PreprocessorTok}[1]{\textcolor[rgb]{0.56,0.35,0.01}{\textit{#1}}}
\newcommand{\RegionMarkerTok}[1]{#1}
\newcommand{\SpecialCharTok}[1]{\textcolor[rgb]{0.81,0.36,0.00}{\textbf{#1}}}
\newcommand{\SpecialStringTok}[1]{\textcolor[rgb]{0.31,0.60,0.02}{#1}}
\newcommand{\StringTok}[1]{\textcolor[rgb]{0.31,0.60,0.02}{#1}}
\newcommand{\VariableTok}[1]{\textcolor[rgb]{0.00,0.00,0.00}{#1}}
\newcommand{\VerbatimStringTok}[1]{\textcolor[rgb]{0.31,0.60,0.02}{#1}}
\newcommand{\WarningTok}[1]{\textcolor[rgb]{0.56,0.35,0.01}{\textbf{\textit{#1}}}}
\usepackage{graphicx}
\makeatletter
\newsavebox\pandoc@box
\newcommand*\pandocbounded[1]{% scales image to fit in text height/width
  \sbox\pandoc@box{#1}%
  \Gscale@div\@tempa{\textheight}{\dimexpr\ht\pandoc@box+\dp\pandoc@box\relax}%
  \Gscale@div\@tempb{\linewidth}{\wd\pandoc@box}%
  \ifdim\@tempb\p@<\@tempa\p@\let\@tempa\@tempb\fi% select the smaller of both
  \ifdim\@tempa\p@<\p@\scalebox{\@tempa}{\usebox\pandoc@box}%
  \else\usebox{\pandoc@box}%
  \fi%
}
% Set default figure placement to htbp
\def\fps@figure{htbp}
\makeatother
\setlength{\emergencystretch}{3em} % prevent overfull lines
\providecommand{\tightlist}{%
  \setlength{\itemsep}{0pt}\setlength{\parskip}{0pt}}
\usepackage{caption}
\captionsetup[table]{justification=centering}
\usepackage{booktabs}
\usepackage{longtable}
\usepackage{array}
\usepackage{multirow}
\usepackage{wrapfig}
\usepackage{float}
\usepackage{colortbl}
\usepackage{pdflscape}
\usepackage{tabu}
\usepackage{threeparttable}
\usepackage{threeparttablex}
\usepackage[normalem]{ulem}
\usepackage{makecell}
\usepackage{xcolor}
\usepackage{fontspec}
\usepackage{multicol}
\usepackage{hhline}
\newlength\Oldarrayrulewidth
\newlength\Oldtabcolsep
\usepackage{hyperref}
\usepackage{bookmark}
\IfFileExists{xurl.sty}{\usepackage{xurl}}{} % add URL line breaks if available
\urlstyle{same}
\hypersetup{
  pdftitle={Thang\_Activity 4},
  hidelinks,
  pdfcreator={LaTeX via pandoc}}

\title{Thang\_Activity 4}
\author{}
\date{\vspace{-2.5em}2025-11-09}

\begin{document}
\maketitle

\begin{Shaded}
\begin{Highlighting}[]
\DocumentationTok{\#\# Set global CRAN mirror}
\FunctionTok{options}\NormalTok{(}\AttributeTok{repos =} \FunctionTok{c}\NormalTok{(}\AttributeTok{CRAN =} \StringTok{"https://cran.rstudio.com/"}\NormalTok{))}

\DocumentationTok{\#\# Install the tableone package}
\FunctionTok{install.packages}\NormalTok{(}\StringTok{"tableone"}\NormalTok{)}
\end{Highlighting}
\end{Shaded}

\begin{verbatim}
## Installing package into 'C:/Users/Thang/AppData/Local/R/win-library/4.3'
## (as 'lib' is unspecified)
\end{verbatim}

\begin{verbatim}
## package 'tableone' successfully unpacked and MD5 sums checked
## 
## The downloaded binary packages are in
##  C:\Users\Thang\AppData\Local\Temp\RtmpATs4He\downloaded_packages
\end{verbatim}

\begin{Shaded}
\begin{Highlighting}[]
\DocumentationTok{\#\# Load the required packages}

\FunctionTok{library}\NormalTok{(tableone)}
\FunctionTok{library}\NormalTok{(readxl)}
\FunctionTok{library}\NormalTok{(dplyr)}
\end{Highlighting}
\end{Shaded}

\begin{verbatim}
## 
## Attaching package: 'dplyr'
\end{verbatim}

\begin{verbatim}
## The following objects are masked from 'package:stats':
## 
##     filter, lag
\end{verbatim}

\begin{verbatim}
## The following objects are masked from 'package:base':
## 
##     intersect, setdiff, setequal, union
\end{verbatim}

\begin{Shaded}
\begin{Highlighting}[]
\FunctionTok{library}\NormalTok{(janitor)}
\end{Highlighting}
\end{Shaded}

\begin{verbatim}
## 
## Attaching package: 'janitor'
\end{verbatim}

\begin{verbatim}
## The following objects are masked from 'package:stats':
## 
##     chisq.test, fisher.test
\end{verbatim}

\begin{Shaded}
\begin{Highlighting}[]
\FunctionTok{library}\NormalTok{(knitr)}

\FunctionTok{setwd}\NormalTok{(}\StringTok{"C:/Users/Thang/OneDrive/Documents/ActivityData\_Assignment/Activity\_4"}\NormalTok{)}
\NormalTok{data }\OtherTok{\textless{}{-}} \FunctionTok{read\_excel}\NormalTok{(}\StringTok{"SuicideRisk\_Data.xlsx"}\NormalTok{)}
\end{Highlighting}
\end{Shaded}

\section{QUESTION 1.}\label{question-1.}

\begin{Shaded}
\begin{Highlighting}[]
\CommentTok{\# ============================}
\CommentTok{\# PART A: SABCS \textasciitilde{} Demographics}
\CommentTok{\# ============================}

\CommentTok{\# Convert categorical variables to factors}
\NormalTok{data}\SpecialCharTok{$}\NormalTok{GENDER    }\OtherTok{\textless{}{-}} \FunctionTok{factor}\NormalTok{(data}\SpecialCharTok{$}\NormalTok{GENDER)}
\NormalTok{data}\SpecialCharTok{$}\NormalTok{RACE      }\OtherTok{\textless{}{-}} \FunctionTok{factor}\NormalTok{(data}\SpecialCharTok{$}\NormalTok{RACE)}
\NormalTok{data}\SpecialCharTok{$}\NormalTok{ETHNICITY }\OtherTok{\textless{}{-}} \FunctionTok{factor}\NormalTok{(data}\SpecialCharTok{$}\NormalTok{ETHNICITY)}
\NormalTok{data}\SpecialCharTok{$}\NormalTok{INCOME    }\OtherTok{\textless{}{-}} \FunctionTok{factor}\NormalTok{(data}\SpecialCharTok{$}\NormalTok{INCOME)}

\NormalTok{model\_a }\OtherTok{\textless{}{-}} \FunctionTok{lm}\NormalTok{(SABCS\_TOTAL\_SUM }\SpecialCharTok{\textasciitilde{}}\NormalTok{ AGE }\SpecialCharTok{+}\NormalTok{ GENDER }\SpecialCharTok{+}\NormalTok{ RACE }\SpecialCharTok{+}\NormalTok{ ETHNICITY }\SpecialCharTok{+}\NormalTok{ INCOME, }\AttributeTok{data =}\NormalTok{ data)}

\FunctionTok{summary}\NormalTok{(model\_a)}
\end{Highlighting}
\end{Shaded}

\begin{verbatim}
## 
## Call:
## lm(formula = SABCS_TOTAL_SUM ~ AGE + GENDER + RACE + ETHNICITY + 
##     INCOME, data = data)
## 
## Residuals:
##     Min      1Q  Median      3Q     Max 
## -6.1136 -2.9988 -1.5688  0.7734 28.8824 
## 
## Coefficients:
##                              Estimate Std. Error t value Pr(>|t|)    
## (Intercept)                   6.26744    1.20338   5.208 2.72e-07 ***
## AGE                          -0.06428    0.03280  -1.960   0.0505 .  
## GENDERMale                   -0.28798    0.74409  -0.387   0.6989    
## RACEOther                    -0.10208    0.81829  -0.125   0.9008    
## RACEWhite/Caucasian          -0.35525    0.67499  -0.526   0.5989    
## ETHNICITYNot Hispanic/Latino -1.18615    0.65198  -1.819   0.0694 .  
## INCOME$51,000 - $75,000      -1.04222    0.67541  -1.543   0.1234    
## INCOME$76,000 - $100,000     -0.31306    0.69360  -0.451   0.6519    
## INCOME< $30,000               1.16960    0.66952   1.747   0.0812 .  
## INCOME>$100,000               0.02945    0.67065   0.044   0.9650    
## ---
## Signif. codes:  0 '***' 0.001 '**' 0.01 '*' 0.05 '.' 0.1 ' ' 1
## 
## Residual standard error: 4.808 on 536 degrees of freedom
## Multiple R-squared:  0.03723,    Adjusted R-squared:  0.02107 
## F-statistic: 2.303 on 9 and 536 DF,  p-value: 0.01522
\end{verbatim}

\begin{Shaded}
\begin{Highlighting}[]
\CommentTok{\# Diagnostics}
\FunctionTok{par}\NormalTok{(}\AttributeTok{mfrow=}\FunctionTok{c}\NormalTok{(}\DecValTok{2}\NormalTok{,}\DecValTok{2}\NormalTok{))}
\FunctionTok{plot}\NormalTok{(model\_a)}
\end{Highlighting}
\end{Shaded}

\pandocbounded{\includegraphics[keepaspectratio]{Activity_4_files/figure-latex/unnamed-chunk-2-1.pdf}}

\begin{Shaded}
\begin{Highlighting}[]
\CommentTok{\# Reset plotting}
\FunctionTok{par}\NormalTok{(}\AttributeTok{mfrow=}\FunctionTok{c}\NormalTok{(}\DecValTok{1}\NormalTok{,}\DecValTok{1}\NormalTok{))}
\end{Highlighting}
\end{Shaded}

The demographic model accounts for a mere 3.7\% of the variation in
suicidal risk, indicating that demographics alone do not serve as robust
predictors of SABCS scores. Age has a small negative correlation,
indicating a somewhat diminished suicide risk among older participants;
nevertheless, the effect is minor. Ethnicity (Not Hispanic/Latino) and
low income (\textless\$30,000) exhibit marginal correlations with SABCS
results, however none serve as robust predictors. The majority of
demographic factors, such as gender and ethnicity, lack statistical
significance. In general, demographic traits don't give us much
information on how likely someone in this group is to commit suicide.

\begin{Shaded}
\begin{Highlighting}[]
\CommentTok{\# ============================}
\CommentTok{\# PART B: SABCS \textasciitilde{} Depression}
\CommentTok{\# ============================}

\NormalTok{model\_b }\OtherTok{\textless{}{-}} \FunctionTok{lm}\NormalTok{(SABCS\_TOTAL\_SUM }\SpecialCharTok{\textasciitilde{}}\NormalTok{ CESDR\_TOTAL\_SUM, }\AttributeTok{data =}\NormalTok{ data)}

\FunctionTok{summary}\NormalTok{(model\_b)}
\end{Highlighting}
\end{Shaded}

\begin{verbatim}
## 
## Call:
## lm(formula = SABCS_TOTAL_SUM ~ CESDR_TOTAL_SUM, data = data)
## 
## Residuals:
##      Min       1Q   Median       3Q      Max 
## -11.2968  -2.1009  -0.4739   0.9891  21.3302 
## 
## Coefficients:
##                 Estimate Std. Error t value Pr(>|t|)    
## (Intercept)      0.01088    0.25650   0.042    0.966    
## CESDR_TOTAL_SUM  0.20900    0.01228  17.013   <2e-16 ***
## ---
## Signif. codes:  0 '***' 0.001 '**' 0.01 '*' 0.05 '.' 0.1 ' ' 1
## 
## Residual standard error: 3.93 on 544 degrees of freedom
## Multiple R-squared:  0.3473, Adjusted R-squared:  0.3461 
## F-statistic: 289.4 on 1 and 544 DF,  p-value: < 2.2e-16
\end{verbatim}

\begin{Shaded}
\begin{Highlighting}[]
\CommentTok{\# Diagnostics}
\FunctionTok{par}\NormalTok{(}\AttributeTok{mfrow=}\FunctionTok{c}\NormalTok{(}\DecValTok{2}\NormalTok{,}\DecValTok{2}\NormalTok{))}
\FunctionTok{plot}\NormalTok{(model\_b)}
\end{Highlighting}
\end{Shaded}

\pandocbounded{\includegraphics[keepaspectratio]{Activity_4_files/figure-latex/unnamed-chunk-3-1.pdf}}

\begin{Shaded}
\begin{Highlighting}[]
\FunctionTok{par}\NormalTok{(}\AttributeTok{mfrow=}\FunctionTok{c}\NormalTok{(}\DecValTok{1}\NormalTok{,}\DecValTok{1}\NormalTok{))}
\end{Highlighting}
\end{Shaded}

There is a substantial and significant positive relationship between
depressive symptoms (CESD-R scores) and suicide risk, which means that
higher depression levels are linked to higher SABCS scores. The model
elucidates around 35\% of the variance in suicide risk, far surpassing
the demographic model. In this sample, depression is a strong indicator
of the probability of suicide.

\begin{Shaded}
\begin{Highlighting}[]
\CommentTok{\# ============================}
\CommentTok{\# PART C: SABCS \textasciitilde{} ACES 1–10}
\CommentTok{\# ============================}

\NormalTok{aces\_vars }\OtherTok{\textless{}{-}} \FunctionTok{paste0}\NormalTok{(}\StringTok{"ACES\_\_\_"}\NormalTok{, }\DecValTok{1}\SpecialCharTok{:}\DecValTok{10}\NormalTok{)}

\NormalTok{model\_c }\OtherTok{\textless{}{-}} \FunctionTok{lm}\NormalTok{(}\FunctionTok{as.formula}\NormalTok{(}
  \FunctionTok{paste}\NormalTok{(}\StringTok{"SABCS\_TOTAL\_SUM \textasciitilde{}"}\NormalTok{, }\FunctionTok{paste}\NormalTok{(aces\_vars, }\AttributeTok{collapse =} \StringTok{" + "}\NormalTok{))}
\NormalTok{), }\AttributeTok{data =}\NormalTok{ data)}

\FunctionTok{summary}\NormalTok{(model\_c)}
\end{Highlighting}
\end{Shaded}

\begin{verbatim}
## 
## Call:
## lm(formula = as.formula(paste("SABCS_TOTAL_SUM ~", paste(aces_vars, 
##     collapse = " + "))), data = data)
## 
## Residuals:
##     Min      1Q  Median      3Q     Max 
## -7.9486 -2.2530 -1.2530  0.9881 24.3547 
## 
## Coefficients:
##             Estimate Std. Error t value Pr(>|t|)    
## (Intercept)  2.25301    0.25704   8.765  < 2e-16 ***
## ACES___1     0.19534    0.72244   0.270 0.786969    
## ACES___2     2.06336    0.56740   3.637 0.000303 ***
## ACES___3     2.23049    0.56574   3.943 9.13e-05 ***
## ACES___4     0.09203    1.26894   0.073 0.942208    
## ACES___5     0.81234    0.70763   1.148 0.251491    
## ACES___6    -0.53680    0.65125  -0.824 0.410164    
## ACES___7     0.57879    0.66378   0.872 0.383619    
## ACES___8     1.63521    0.53899   3.034 0.002532 ** 
## ACES___9    -1.24110    0.47201  -2.629 0.008799 ** 
## ACES___10   -1.44145    0.99893  -1.443 0.149608    
## ---
## Signif. codes:  0 '***' 0.001 '**' 0.01 '*' 0.05 '.' 0.1 ' ' 1
## 
## Residual standard error: 4.533 on 535 degrees of freedom
## Multiple R-squared:  0.146,  Adjusted R-squared:   0.13 
## F-statistic: 9.147 on 10 and 535 DF,  p-value: 5.074e-14
\end{verbatim}

\begin{Shaded}
\begin{Highlighting}[]
\CommentTok{\# Diagnostics}
\FunctionTok{par}\NormalTok{(}\AttributeTok{mfrow=}\FunctionTok{c}\NormalTok{(}\DecValTok{2}\NormalTok{,}\DecValTok{2}\NormalTok{))}
\FunctionTok{plot}\NormalTok{(model\_c)}
\end{Highlighting}
\end{Shaded}

\pandocbounded{\includegraphics[keepaspectratio]{Activity_4_files/figure-latex/unnamed-chunk-4-1.pdf}}

\begin{Shaded}
\begin{Highlighting}[]
\FunctionTok{par}\NormalTok{(}\AttributeTok{mfrow=}\FunctionTok{c}\NormalTok{(}\DecValTok{1}\NormalTok{,}\DecValTok{1}\NormalTok{))}

\CommentTok{\# Extract coefficients with p \textless{} 0.20}
\NormalTok{get\_sig }\OtherTok{\textless{}{-}} \ControlFlowTok{function}\NormalTok{(model, original\_vars) \{}
\NormalTok{  tbl }\OtherTok{\textless{}{-}} \FunctionTok{summary}\NormalTok{(model)}\SpecialCharTok{$}\NormalTok{coefficients}
\NormalTok{  sig\_terms }\OtherTok{\textless{}{-}} \FunctionTok{rownames}\NormalTok{(tbl)[tbl[,}\DecValTok{4}\NormalTok{] }\SpecialCharTok{\textless{}} \FloatTok{0.20}\NormalTok{]   }\CommentTok{\# significant rows (dummy names)}
\NormalTok{  sig\_terms }\OtherTok{\textless{}{-}}\NormalTok{ sig\_terms[sig\_terms }\SpecialCharTok{!=} \StringTok{"(Intercept)"}\NormalTok{]}

\NormalTok{  matched }\OtherTok{\textless{}{-}} \FunctionTok{c}\NormalTok{()}
  \ControlFlowTok{for}\NormalTok{ (term }\ControlFlowTok{in}\NormalTok{ sig\_terms) \{}
\NormalTok{    base }\OtherTok{\textless{}{-}}\NormalTok{ original\_vars[}\FunctionTok{sapply}\NormalTok{(original\_vars, }\ControlFlowTok{function}\NormalTok{(v) }\FunctionTok{startsWith}\NormalTok{(term, v))]}
\NormalTok{    matched }\OtherTok{\textless{}{-}} \FunctionTok{c}\NormalTok{(matched, base)}
\NormalTok{  \}}

  \FunctionTok{unique}\NormalTok{(matched)}
\NormalTok{\}}
\NormalTok{orig\_a }\OtherTok{\textless{}{-}} \FunctionTok{c}\NormalTok{(}\StringTok{"AGE"}\NormalTok{, }\StringTok{"GENDER"}\NormalTok{, }\StringTok{"RACE"}\NormalTok{, }\StringTok{"ETHNICITY"}\NormalTok{, }\StringTok{"INCOME"}\NormalTok{)}
\NormalTok{orig\_b }\OtherTok{\textless{}{-}} \FunctionTok{c}\NormalTok{(}\StringTok{"CESDR\_TOTAL\_SUM"}\NormalTok{)}
\NormalTok{orig\_c }\OtherTok{\textless{}{-}} \FunctionTok{paste0}\NormalTok{(}\StringTok{"ACES\_\_\_"}\NormalTok{, }\DecValTok{1}\SpecialCharTok{:}\DecValTok{10}\NormalTok{)}

\NormalTok{sig\_a }\OtherTok{\textless{}{-}} \FunctionTok{get\_sig}\NormalTok{(model\_a, orig\_a)}
\NormalTok{sig\_b }\OtherTok{\textless{}{-}} \FunctionTok{get\_sig}\NormalTok{(model\_b, orig\_b)}
\NormalTok{sig\_c }\OtherTok{\textless{}{-}} \FunctionTok{get\_sig}\NormalTok{(model\_c, orig\_c)}

\NormalTok{sig\_a; sig\_b; sig\_c}
\end{Highlighting}
\end{Shaded}

\begin{verbatim}
## [1] "AGE"       "ETHNICITY" "INCOME"
\end{verbatim}

\begin{verbatim}
## [1] "CESDR_TOTAL_SUM"
\end{verbatim}

\begin{verbatim}
## [1] "ACES___2"  "ACES___3"  "ACES___8"  "ACES___9"  "ACES___1"  "ACES___10"
\end{verbatim}

Several ACE categories exhibit significant correlations with suicide
risk, indicating that specific forms of childhood adversity are closely
linked to elevated SABCS scores. In particular, ACEs 2, 3, and 8 are
connected to a much increased risk of suicide, while ACE 9 shows a
negative link. Most other ACE items do not have significant impacts,
indicating that not all forms of childhood adversity equally influence
suicidal thoughts or behaviors. The model accounts for around 14--15\%
of the variation in suicide risk, suggesting that childhood events are
significant, though not the sole factor. Overall, ACEs help predict the
risk of suicide, although the effects depend on the sort of adversity.

\begin{Shaded}
\begin{Highlighting}[]
\CommentTok{\# ============================}
\CommentTok{\# PART D}
\CommentTok{\# ============================}
\NormalTok{all\_sig }\OtherTok{\textless{}{-}} \FunctionTok{unique}\NormalTok{(}\FunctionTok{c}\NormalTok{(sig\_a, sig\_b, sig\_c))}

\NormalTok{formula\_d }\OtherTok{\textless{}{-}} \FunctionTok{as.formula}\NormalTok{(}
  \FunctionTok{paste}\NormalTok{(}\StringTok{"SABCS\_TOTAL\_SUM \textasciitilde{}"}\NormalTok{, }\FunctionTok{paste}\NormalTok{(all\_sig, }\AttributeTok{collapse =} \StringTok{" + "}\NormalTok{))}
\NormalTok{)}

\NormalTok{model\_d }\OtherTok{\textless{}{-}} \FunctionTok{lm}\NormalTok{(formula\_d, }\AttributeTok{data =}\NormalTok{ data)}
\FunctionTok{summary}\NormalTok{(model\_d)}
\end{Highlighting}
\end{Shaded}

\begin{verbatim}
## 
## Call:
## lm(formula = formula_d, data = data)
## 
## Residuals:
##      Min       1Q   Median       3Q      Max 
## -11.6914  -1.9593  -0.4288   1.2998  18.4356 
## 
## Coefficients:
##                              Estimate Std. Error t value Pr(>|t|)    
## (Intercept)                   1.89044    0.90389   2.091  0.03696 *  
## AGE                          -0.05315    0.02637  -2.016  0.04432 *  
## ETHNICITYNot Hispanic/Latino -1.00572    0.50380  -1.996  0.04641 *  
## INCOME$51,000 - $75,000      -0.44277    0.52894  -0.837  0.40292    
## INCOME$76,000 - $100,000      0.20055    0.53890   0.372  0.70993    
## INCOME< $30,000               0.57864    0.52733   1.097  0.27301    
## INCOME>$100,000               0.78980    0.51999   1.519  0.12939    
## CESDR_TOTAL_SUM               0.18484    0.01269  14.560  < 2e-16 ***
## ACES___2                      1.36291    0.47280   2.883  0.00410 ** 
## ACES___3                      1.38402    0.45800   3.022  0.00263 ** 
## ACES___8                      0.83230    0.44182   1.884  0.06014 .  
## ACES___9                     -1.27265    0.39121  -3.253  0.00121 ** 
## ACES___1                      0.11834    0.59304   0.200  0.84191    
## ACES___10                    -0.21860    0.80080  -0.273  0.78498    
## ---
## Signif. codes:  0 '***' 0.001 '**' 0.01 '*' 0.05 '.' 0.1 ' ' 1
## 
## Residual standard error: 3.772 on 532 degrees of freedom
## Multiple R-squared:  0.4117, Adjusted R-squared:  0.3973 
## F-statistic: 28.64 on 13 and 532 DF,  p-value: < 2.2e-16
\end{verbatim}

\begin{Shaded}
\begin{Highlighting}[]
\FunctionTok{par}\NormalTok{(}\AttributeTok{mfrow=}\FunctionTok{c}\NormalTok{(}\DecValTok{2}\NormalTok{,}\DecValTok{2}\NormalTok{)); }\FunctionTok{plot}\NormalTok{(model\_d)}
\end{Highlighting}
\end{Shaded}

\pandocbounded{\includegraphics[keepaspectratio]{Activity_4_files/figure-latex/unnamed-chunk-5-1.pdf}}
The combined model elucidates approximately 41\% of the heterogeneity in
suicide risk, representing a significant enhancement over the separate
demographic and ACE models. Depression (CESDR scores) continues to be
the most significant predictor, even after controlling for demographics
and childhood adversity, demonstrating its critical role in suicidal
risk. A number of ACE categories (ACES 2, 3, and 9) continue to exhibit
substantial correlations, indicating that particular childhood events
still have major implications beyond depression. Age and ethnicity
exhibit minor yet statistically significant effects, whereas wealth
demonstrates less influence. In general, integrating demographics,
depression, and ACEs gives a better picture of the risk of suicide.

\begin{Shaded}
\begin{Highlighting}[]
\CommentTok{\# ============================}
\CommentTok{\# PART E: sqrt(SABCS) model}
\CommentTok{\# ============================}

\CommentTok{\# Create transformed variable}
\NormalTok{data}\SpecialCharTok{$}\NormalTok{SABCS\_sqrt }\OtherTok{\textless{}{-}} \FunctionTok{sqrt}\NormalTok{(data}\SpecialCharTok{$}\NormalTok{SABCS\_TOTAL\_SUM)}

\CommentTok{\# Build formula using same predictors as Part D}
\NormalTok{formula\_e }\OtherTok{\textless{}{-}} \FunctionTok{as.formula}\NormalTok{(}
  \FunctionTok{paste}\NormalTok{(}\StringTok{"SABCS\_sqrt \textasciitilde{}"}\NormalTok{, }\FunctionTok{paste}\NormalTok{(all\_sig, }\AttributeTok{collapse =} \StringTok{" + "}\NormalTok{))}
\NormalTok{)}

\CommentTok{\# Fit model}
\NormalTok{model\_e }\OtherTok{\textless{}{-}} \FunctionTok{lm}\NormalTok{(formula\_e, }\AttributeTok{data =}\NormalTok{ data)}

\CommentTok{\# Show results}
\FunctionTok{summary}\NormalTok{(model\_e)}
\end{Highlighting}
\end{Shaded}

\begin{verbatim}
## 
## Call:
## lm(formula = formula_e, data = data)
## 
## Residuals:
##     Min      1Q  Median      3Q     Max 
## -3.4442 -0.6833 -0.0962  0.6821  2.8227 
## 
## Coefficients:
##                               Estimate Std. Error t value Pr(>|t|)    
## (Intercept)                   0.908047   0.240276   3.779 0.000175 ***
## AGE                          -0.012019   0.007010  -1.715 0.086982 .  
## ETHNICITYNot Hispanic/Latino -0.263033   0.133921  -1.964 0.050040 .  
## INCOME$51,000 - $75,000      -0.218214   0.140606  -1.552 0.121268    
## INCOME$76,000 - $100,000      0.041328   0.143254   0.288 0.773081    
## INCOME< $30,000               0.131719   0.140176   0.940 0.347813    
## INCOME>$100,000               0.147564   0.138227   1.068 0.286208    
## CESDR_TOTAL_SUM               0.047502   0.003375  14.076  < 2e-16 ***
## ACES___2                      0.354801   0.125681   2.823 0.004935 ** 
## ACES___3                      0.331545   0.121747   2.723 0.006677 ** 
## ACES___8                      0.231329   0.117447   1.970 0.049397 *  
## ACES___9                     -0.280136   0.103993  -2.694 0.007287 ** 
## ACES___1                      0.059622   0.157644   0.378 0.705425    
## ACES___10                     0.120804   0.212873   0.567 0.570617    
## ---
## Signif. codes:  0 '***' 0.001 '**' 0.01 '*' 0.05 '.' 0.1 ' ' 1
## 
## Residual standard error: 1.003 on 532 degrees of freedom
## Multiple R-squared:    0.4,  Adjusted R-squared:  0.3853 
## F-statistic: 27.28 on 13 and 532 DF,  p-value: < 2.2e-16
\end{verbatim}

\begin{Shaded}
\begin{Highlighting}[]
\CommentTok{\# Diagnostics}
\FunctionTok{par}\NormalTok{(}\AttributeTok{mfrow=}\FunctionTok{c}\NormalTok{(}\DecValTok{2}\NormalTok{,}\DecValTok{2}\NormalTok{))}
\FunctionTok{plot}\NormalTok{(model\_e)}
\end{Highlighting}
\end{Shaded}

\pandocbounded{\includegraphics[keepaspectratio]{Activity_4_files/figure-latex/unnamed-chunk-6-1.pdf}}

\begin{Shaded}
\begin{Highlighting}[]
\FunctionTok{par}\NormalTok{(}\AttributeTok{mfrow=}\FunctionTok{c}\NormalTok{(}\DecValTok{1}\NormalTok{,}\DecValTok{1}\NormalTok{))}
\end{Highlighting}
\end{Shaded}

Following the square-root transformation, the model exhibits analogous
patterns of important predictors as shown in Part D, with depression
persisting as the most substantial factor linked to suicide risk. ACE
items 2, 3, 8, and 9 still have significant effects, but the sizes of
the effects are reduced because the transformation has made the outcome
smaller. The residuals are more evenly distributed and have a smaller
range, which means that the transformation makes the data more normal
and stabilizes the variance. The converted model fits the data better
overall, while still keeping the primary associations that were shown in
the untransformed model.

\begin{Shaded}
\begin{Highlighting}[]
\CommentTok{\# ============================}
\CommentTok{\# PART F: Compare Model D vs E}
\CommentTok{\# ============================}

\CommentTok{\# Compare AIC \& BIC}
\FunctionTok{AIC}\NormalTok{(model\_d, model\_e)}
\end{Highlighting}
\end{Shaded}

\begin{verbatim}
##         df      AIC
## model_d 15 3015.188
## model_e 15 1568.376
\end{verbatim}

\begin{Shaded}
\begin{Highlighting}[]
\FunctionTok{BIC}\NormalTok{(model\_d, model\_e)}
\end{Highlighting}
\end{Shaded}

\begin{verbatim}
##         df      BIC
## model_d 15 3079.727
## model_e 15 1632.915
\end{verbatim}

\begin{Shaded}
\begin{Highlighting}[]
\CommentTok{\# Residual histograms}
\FunctionTok{par}\NormalTok{(}\AttributeTok{mfrow=}\FunctionTok{c}\NormalTok{(}\DecValTok{1}\NormalTok{,}\DecValTok{2}\NormalTok{))}
\FunctionTok{hist}\NormalTok{(}\FunctionTok{resid}\NormalTok{(model\_d), }\AttributeTok{main=}\StringTok{"Residuals: Model D"}\NormalTok{, }\AttributeTok{xlab=}\StringTok{""}\NormalTok{, }\AttributeTok{col=}\StringTok{"lightblue"}\NormalTok{)}
\FunctionTok{hist}\NormalTok{(}\FunctionTok{resid}\NormalTok{(model\_e), }\AttributeTok{main=}\StringTok{"Residuals: Model E (sqrt)"}\NormalTok{, }\AttributeTok{xlab=}\StringTok{""}\NormalTok{, }\AttributeTok{col=}\StringTok{"lightgreen"}\NormalTok{)}
\end{Highlighting}
\end{Shaded}

\pandocbounded{\includegraphics[keepaspectratio]{Activity_4_files/figure-latex/unnamed-chunk-7-1.pdf}}

\begin{Shaded}
\begin{Highlighting}[]
\FunctionTok{par}\NormalTok{(}\AttributeTok{mfrow=}\FunctionTok{c}\NormalTok{(}\DecValTok{1}\NormalTok{,}\DecValTok{1}\NormalTok{))}

\CommentTok{\# Shapiro{-}Wilk normality test}
\FunctionTok{shapiro.test}\NormalTok{(}\FunctionTok{resid}\NormalTok{(model\_d))}
\end{Highlighting}
\end{Shaded}

\begin{verbatim}
## 
##  Shapiro-Wilk normality test
## 
## data:  resid(model_d)
## W = 0.93427, p-value = 9.056e-15
\end{verbatim}

\begin{Shaded}
\begin{Highlighting}[]
\FunctionTok{shapiro.test}\NormalTok{(}\FunctionTok{resid}\NormalTok{(model\_e))}
\end{Highlighting}
\end{Shaded}

\begin{verbatim}
## 
##  Shapiro-Wilk normality test
## 
## data:  resid(model_e)
## W = 0.99344, p-value = 0.01771
\end{verbatim}

\begin{Shaded}
\begin{Highlighting}[]
\CommentTok{\# Residual vs Fitted comparison}
\FunctionTok{par}\NormalTok{(}\AttributeTok{mfrow=}\FunctionTok{c}\NormalTok{(}\DecValTok{1}\NormalTok{,}\DecValTok{2}\NormalTok{))}
\FunctionTok{plot}\NormalTok{(model\_d}\SpecialCharTok{$}\NormalTok{fitted.values, }\FunctionTok{resid}\NormalTok{(model\_d),}
     \AttributeTok{main=}\StringTok{"Model D Residual vs Fitted"}\NormalTok{,}
     \AttributeTok{xlab=}\StringTok{"Fitted"}\NormalTok{, }\AttributeTok{ylab=}\StringTok{"Residuals"}\NormalTok{)}

\FunctionTok{plot}\NormalTok{(model\_e}\SpecialCharTok{$}\NormalTok{fitted.values, }\FunctionTok{resid}\NormalTok{(model\_e),}
     \AttributeTok{main=}\StringTok{"Model E Residual vs Fitted"}\NormalTok{,}
     \AttributeTok{xlab=}\StringTok{"Fitted"}\NormalTok{, }\AttributeTok{ylab=}\StringTok{"Residuals"}\NormalTok{)}
\end{Highlighting}
\end{Shaded}

\pandocbounded{\includegraphics[keepaspectratio]{Activity_4_files/figure-latex/unnamed-chunk-7-2.pdf}}

\begin{Shaded}
\begin{Highlighting}[]
\FunctionTok{par}\NormalTok{(}\AttributeTok{mfrow=}\FunctionTok{c}\NormalTok{(}\DecValTok{1}\NormalTok{,}\DecValTok{1}\NormalTok{))}
\end{Highlighting}
\end{Shaded}

The square-root-transformed model (Model E) has much lower AIC and BIC
values than the untransformed model (Model D). This means that it fits
the data better. The Shapiro--Wilk test also shows that the residuals
from Model E are substantially closer to a normal distribution (p ≈
0.018) than those from Model D (p \textless{} 0.00000000000001). This
means that Model E is more normal. Even though the transformed model is
harder to understand because the coefficients are based on the square
root of suicide risk instead of the raw score, it fits the statistical
assumptions better. Taking into account the trade-off between
interpretability and statistical validity, Model E is a stronger choice
because it fits better and is more trustworthy.

\section{QUESTION 2.}\label{question-2.}

\begin{Shaded}
\begin{Highlighting}[]
\CommentTok{\# ============================}
\CommentTok{\# PART A: HX\_SUICIDE \textasciitilde{} Demographics}
\CommentTok{\# ============================}

\NormalTok{model\_a\_hx }\OtherTok{\textless{}{-}} \FunctionTok{lm}\NormalTok{(HX\_SUICIDE }\SpecialCharTok{\textasciitilde{}}\NormalTok{ AGE }\SpecialCharTok{+}\NormalTok{ GENDER }\SpecialCharTok{+}\NormalTok{ RACE }\SpecialCharTok{+}\NormalTok{ ETHNICITY }\SpecialCharTok{+}\NormalTok{ INCOME, }\AttributeTok{data =}\NormalTok{ data)}

\FunctionTok{summary}\NormalTok{(model\_a\_hx)}
\end{Highlighting}
\end{Shaded}

\begin{verbatim}
## 
## Call:
## lm(formula = HX_SUICIDE ~ AGE + GENDER + RACE + ETHNICITY + INCOME, 
##     data = data)
## 
## Residuals:
##      Min       1Q   Median       3Q      Max 
## -0.21551 -0.10245 -0.07885 -0.05568  0.95821 
## 
## Coefficients:
##                               Estimate Std. Error t value Pr(>|t|)  
## (Intercept)                   0.138268   0.071498   1.934   0.0537 .
## AGE                           0.001672   0.001949   0.858   0.3912  
## GENDERMale                    0.012110   0.044210   0.274   0.7843  
## RACEOther                    -0.038807   0.048618  -0.798   0.4251  
## RACEWhite/Caucasian           0.017234   0.040104   0.430   0.6676  
## ETHNICITYNot Hispanic/Latino -0.089846   0.038737  -2.319   0.0207 *
## INCOME$51,000 - $75,000      -0.045096   0.040129  -1.124   0.2616  
## INCOME$76,000 - $100,000     -0.029880   0.041210  -0.725   0.4687  
## INCOME< $30,000               0.006487   0.039779   0.163   0.8705  
## INCOME>$100,000              -0.021926   0.039846  -0.550   0.5824  
## ---
## Signif. codes:  0 '***' 0.001 '**' 0.01 '*' 0.05 '.' 0.1 ' ' 1
## 
## Residual standard error: 0.2857 on 536 degrees of freedom
## Multiple R-squared:  0.01933,    Adjusted R-squared:  0.002864 
## F-statistic: 1.174 on 9 and 536 DF,  p-value: 0.3092
\end{verbatim}

\begin{Shaded}
\begin{Highlighting}[]
\CommentTok{\# Diagnostics}
\FunctionTok{par}\NormalTok{(}\AttributeTok{mfrow=}\FunctionTok{c}\NormalTok{(}\DecValTok{2}\NormalTok{,}\DecValTok{2}\NormalTok{))}
\FunctionTok{plot}\NormalTok{(model\_a\_hx)}
\end{Highlighting}
\end{Shaded}

\pandocbounded{\includegraphics[keepaspectratio]{Activity_4_files/figure-latex/unnamed-chunk-8-1.pdf}}

\begin{Shaded}
\begin{Highlighting}[]
\FunctionTok{par}\NormalTok{(}\AttributeTok{mfrow=}\FunctionTok{c}\NormalTok{(}\DecValTok{1}\NormalTok{,}\DecValTok{1}\NormalTok{))}
\end{Highlighting}
\end{Shaded}

The demographic model elucidates a mere 2\% of the diversity in the
history of suicide, as indicated by a R² value of approximately 2\%. Of
all the demographic characteristics, only ethnicity (Not
Hispanic/Latino) had a statistically significant link. People in this
category had slightly lower suicide history scores. Other demographic
factors, such as age, gender, ethnicity, and wealth, did not
significantly forecast suicide history. In general, demographic traits
by themselves are not good indicators of whether someone has ever
thought about or attempted suicide.

\begin{Shaded}
\begin{Highlighting}[]
\CommentTok{\# ============================}
\CommentTok{\# PART B: HX\_SUICIDE \textasciitilde{} Depression}
\CommentTok{\# ============================}

\NormalTok{model\_b\_hx }\OtherTok{\textless{}{-}} \FunctionTok{lm}\NormalTok{(HX\_SUICIDE }\SpecialCharTok{\textasciitilde{}}\NormalTok{ CESDR\_TOTAL\_SUM, }\AttributeTok{data =}\NormalTok{ data)}

\FunctionTok{summary}\NormalTok{(model\_b\_hx)}
\end{Highlighting}
\end{Shaded}

\begin{verbatim}
## 
## Call:
## lm(formula = HX_SUICIDE ~ CESDR_TOTAL_SUM, data = data)
## 
## Residuals:
##      Min       1Q   Median       3Q      Max 
## -0.34112 -0.10812 -0.05698 -0.01720  0.98848 
## 
## Coefficients:
##                  Estimate Std. Error t value Pr(>|t|)    
## (Intercept)     0.0001492  0.0179847   0.008    0.993    
## CESDR_TOTAL_SUM 0.0056829  0.0008614   6.598 9.92e-11 ***
## ---
## Signif. codes:  0 '***' 0.001 '**' 0.01 '*' 0.05 '.' 0.1 ' ' 1
## 
## Residual standard error: 0.2755 on 544 degrees of freedom
## Multiple R-squared:  0.07409,    Adjusted R-squared:  0.07239 
## F-statistic: 43.53 on 1 and 544 DF,  p-value: 9.92e-11
\end{verbatim}

\begin{Shaded}
\begin{Highlighting}[]
\CommentTok{\# Diagnostics}
\FunctionTok{par}\NormalTok{(}\AttributeTok{mfrow=}\FunctionTok{c}\NormalTok{(}\DecValTok{2}\NormalTok{,}\DecValTok{2}\NormalTok{))}
\FunctionTok{plot}\NormalTok{(model\_b\_hx)}
\end{Highlighting}
\end{Shaded}

\pandocbounded{\includegraphics[keepaspectratio]{Activity_4_files/figure-latex/unnamed-chunk-9-1.pdf}}

\begin{Shaded}
\begin{Highlighting}[]
\FunctionTok{par}\NormalTok{(}\AttributeTok{mfrow=}\FunctionTok{c}\NormalTok{(}\DecValTok{1}\NormalTok{,}\DecValTok{1}\NormalTok{))}
\end{Highlighting}
\end{Shaded}

Depression exhibits a substantial positive correlation with suicide
history, suggesting that persons with elevated depressive symptoms are
more inclined to disclose a previous suicide attempt or intention. The
overall R² is only about 7\%, but this is still a big step up from the
demographic-only model. These results indicate that depression is a
significant psychological factor associated with suicide history, even
when analyzed in isolation. The diagnostic plots show that the residual
patterns are rather consistent, but the outcome distribution is still
not quite normal because it is a binary-like variable.

\begin{Shaded}
\begin{Highlighting}[]
\CommentTok{\# ============================}
\CommentTok{\# PART C: HX\_SUICIDE \textasciitilde{} ACES 1–10}
\CommentTok{\# ============================}

\NormalTok{aces\_vars }\OtherTok{\textless{}{-}} \FunctionTok{paste0}\NormalTok{(}\StringTok{"ACES\_\_\_"}\NormalTok{, }\DecValTok{1}\SpecialCharTok{:}\DecValTok{10}\NormalTok{)}

\NormalTok{model\_c\_hx }\OtherTok{\textless{}{-}} \FunctionTok{lm}\NormalTok{(}\FunctionTok{as.formula}\NormalTok{(}
  \FunctionTok{paste}\NormalTok{(}\StringTok{"HX\_SUICIDE \textasciitilde{}"}\NormalTok{, }\FunctionTok{paste}\NormalTok{(aces\_vars, }\AttributeTok{collapse =} \StringTok{" + "}\NormalTok{))}
\NormalTok{), }\AttributeTok{data =}\NormalTok{ data)}

\FunctionTok{summary}\NormalTok{(model\_c\_hx)}
\end{Highlighting}
\end{Shaded}

\begin{verbatim}
## 
## Call:
## lm(formula = as.formula(paste("HX_SUICIDE ~", paste(aces_vars, 
##     collapse = " + "))), data = data)
## 
## Residuals:
##      Min       1Q   Median       3Q      Max 
## -0.41211 -0.10457 -0.01968 -0.01968  0.98032 
## 
## Coefficients:
##              Estimate Std. Error t value Pr(>|t|)    
## (Intercept)  0.019678   0.015225   1.293 0.196738    
## ACES___1    -0.019562   0.042791  -0.457 0.647742    
## ACES___2     0.184569   0.033608   5.492 6.16e-08 ***
## ACES___3     0.083640   0.033510   2.496 0.012860 *  
## ACES___4     0.006125   0.075162   0.081 0.935081    
## ACES___5    -0.017496   0.041914  -0.417 0.676541    
## ACES___6    -0.045473   0.038575  -1.179 0.238993    
## ACES___7    -0.057519   0.039317  -1.463 0.144066    
## ACES___8     0.107913   0.031925   3.380 0.000777 ***
## ACES___9     0.016308   0.027958   0.583 0.559940    
## ACES___10    0.100126   0.059169   1.692 0.091189 .  
## ---
## Signif. codes:  0 '***' 0.001 '**' 0.01 '*' 0.05 '.' 0.1 ' ' 1
## 
## Residual standard error: 0.2685 on 535 degrees of freedom
## Multiple R-squared:  0.1355, Adjusted R-squared:  0.1193 
## F-statistic: 8.383 on 10 and 535 DF,  p-value: 1.011e-12
\end{verbatim}

\begin{Shaded}
\begin{Highlighting}[]
\CommentTok{\# Diagnostics}
\FunctionTok{par}\NormalTok{(}\AttributeTok{mfrow=}\FunctionTok{c}\NormalTok{(}\DecValTok{2}\NormalTok{,}\DecValTok{2}\NormalTok{))}
\FunctionTok{plot}\NormalTok{(model\_c\_hx)}
\end{Highlighting}
\end{Shaded}

\pandocbounded{\includegraphics[keepaspectratio]{Activity_4_files/figure-latex/unnamed-chunk-10-1.pdf}}

\begin{Shaded}
\begin{Highlighting}[]
\FunctionTok{par}\NormalTok{(}\AttributeTok{mfrow=}\FunctionTok{c}\NormalTok{(}\DecValTok{1}\NormalTok{,}\DecValTok{1}\NormalTok{))}
\end{Highlighting}
\end{Shaded}

Multiple ACE items were significantly associated with suicide history,
including ACEs 2, 3, and 8. This indicates that certain types of
childhood adversity---particularly emotional, physical, or
household-related trauma---are linked to a higher likelihood of past
suicidal behaviors. The model explained about 13\% of the variation,
suggesting that ACEs play a meaningful but not exclusive role in
predicting suicide history. Several ACE items such as ACE 1, 4, and 5
were not significant, showing that not all childhood experiences
contribute equally.

\begin{Shaded}
\begin{Highlighting}[]
\CommentTok{\# ============================}
\CommentTok{\# PART D}
\CommentTok{\# ============================}
\NormalTok{orig\_dem }\OtherTok{\textless{}{-}} \FunctionTok{c}\NormalTok{(}\StringTok{"AGE"}\NormalTok{, }\StringTok{"GENDER"}\NormalTok{, }\StringTok{"RACE"}\NormalTok{, }\StringTok{"ETHNICITY"}\NormalTok{, }\StringTok{"INCOME"}\NormalTok{)}
\NormalTok{orig\_dep }\OtherTok{\textless{}{-}} \FunctionTok{c}\NormalTok{(}\StringTok{"CESDR\_TOTAL\_SUM"}\NormalTok{)}
\NormalTok{orig\_aces }\OtherTok{\textless{}{-}} \FunctionTok{paste0}\NormalTok{(}\StringTok{"ACES\_\_\_"}\NormalTok{, }\DecValTok{1}\SpecialCharTok{:}\DecValTok{10}\NormalTok{)}
\NormalTok{get\_sig }\OtherTok{\textless{}{-}} \ControlFlowTok{function}\NormalTok{(model, original\_vars) \{}
\NormalTok{  tbl }\OtherTok{\textless{}{-}} \FunctionTok{summary}\NormalTok{(model)}\SpecialCharTok{$}\NormalTok{coefficients}
\NormalTok{  sig\_terms }\OtherTok{\textless{}{-}} \FunctionTok{rownames}\NormalTok{(tbl)[tbl[,}\DecValTok{4}\NormalTok{] }\SpecialCharTok{\textless{}} \FloatTok{0.20}\NormalTok{]}
\NormalTok{  sig\_terms }\OtherTok{\textless{}{-}}\NormalTok{ sig\_terms[sig\_terms }\SpecialCharTok{!=} \StringTok{"(Intercept)"}\NormalTok{]}

\NormalTok{  matched }\OtherTok{\textless{}{-}} \FunctionTok{c}\NormalTok{()}
  \ControlFlowTok{for}\NormalTok{ (term }\ControlFlowTok{in}\NormalTok{ sig\_terms) \{}
\NormalTok{    base }\OtherTok{\textless{}{-}}\NormalTok{ original\_vars[}\FunctionTok{sapply}\NormalTok{(original\_vars, }\ControlFlowTok{function}\NormalTok{(v) }\FunctionTok{startsWith}\NormalTok{(term, v))]}
\NormalTok{    matched }\OtherTok{\textless{}{-}} \FunctionTok{c}\NormalTok{(matched, base)}
\NormalTok{  \}}

  \FunctionTok{unique}\NormalTok{(matched)}
\NormalTok{\}}
\NormalTok{sig\_dem }\OtherTok{\textless{}{-}} \FunctionTok{get\_sig}\NormalTok{(model\_a\_hx, orig\_dem)}
\NormalTok{sig\_dep }\OtherTok{\textless{}{-}} \FunctionTok{get\_sig}\NormalTok{(model\_b\_hx, orig\_dep)}
\NormalTok{sig\_aces }\OtherTok{\textless{}{-}} \FunctionTok{get\_sig}\NormalTok{(model\_c\_hx, orig\_aces)}

\NormalTok{sig\_dem; sig\_dep; sig\_aces}
\end{Highlighting}
\end{Shaded}

\begin{verbatim}
## [1] "ETHNICITY"
\end{verbatim}

\begin{verbatim}
## [1] "CESDR_TOTAL_SUM"
\end{verbatim}

\begin{verbatim}
## [1] "ACES___2"  "ACES___3"  "ACES___7"  "ACES___8"  "ACES___1"  "ACES___10"
\end{verbatim}

\begin{Shaded}
\begin{Highlighting}[]
\CommentTok{\# Combine all predictors at α = 0.20}
\NormalTok{all\_sig\_hx }\OtherTok{\textless{}{-}} \FunctionTok{unique}\NormalTok{(}\FunctionTok{c}\NormalTok{(sig\_dem, sig\_dep, sig\_aces))}
\NormalTok{all\_sig\_hx}
\end{Highlighting}
\end{Shaded}

\begin{verbatim}
## [1] "ETHNICITY"       "CESDR_TOTAL_SUM" "ACES___2"        "ACES___3"       
## [5] "ACES___7"        "ACES___8"        "ACES___1"        "ACES___10"
\end{verbatim}

\begin{Shaded}
\begin{Highlighting}[]
\NormalTok{formula\_hx\_final }\OtherTok{\textless{}{-}} \FunctionTok{as.formula}\NormalTok{(}
  \FunctionTok{paste}\NormalTok{(}\StringTok{"HX\_SUICIDE \textasciitilde{}"}\NormalTok{, }\FunctionTok{paste}\NormalTok{(all\_sig\_hx, }\AttributeTok{collapse =} \StringTok{" + "}\NormalTok{))}
\NormalTok{)}

\NormalTok{model\_hx\_final }\OtherTok{\textless{}{-}} \FunctionTok{lm}\NormalTok{(formula\_hx\_final, }\AttributeTok{data =}\NormalTok{ data)}

\FunctionTok{summary}\NormalTok{(model\_hx\_final)}
\end{Highlighting}
\end{Shaded}

\begin{verbatim}
## 
## Call:
## lm(formula = formula_hx_final, data = data)
## 
## Residuals:
##      Min       1Q   Median       3Q      Max 
## -0.51035 -0.12176 -0.02784  0.02459  1.01653 
## 
## Coefficients:
##                                Estimate Std. Error t value Pr(>|t|)    
## (Intercept)                   0.0333775  0.0355485   0.939  0.34819    
## ETHNICITYNot Hispanic/Latino -0.0700702  0.0343779  -2.038  0.04202 *  
## CESDR_TOTAL_SUM               0.0040330  0.0008685   4.644 4.31e-06 ***
## ACES___2                      0.1656330  0.0327465   5.058 5.82e-07 ***
## ACES___3                      0.0453193  0.0313620   1.445  0.14903    
## ACES___7                     -0.0758090  0.0371180  -2.042  0.04160 *  
## ACES___8                      0.0899596  0.0312015   2.883  0.00409 ** 
## ACES___1                     -0.0253719  0.0404691  -0.627  0.53096    
## ACES___10                     0.1254923  0.0565180   2.220  0.02681 *  
## ---
## Signif. codes:  0 '***' 0.001 '**' 0.01 '*' 0.05 '.' 0.1 ' ' 1
## 
## Residual standard error: 0.2621 on 537 degrees of freedom
## Multiple R-squared:  0.1729, Adjusted R-squared:  0.1606 
## F-statistic: 14.03 on 8 and 537 DF,  p-value: < 2.2e-16
\end{verbatim}

\begin{Shaded}
\begin{Highlighting}[]
\FunctionTok{library}\NormalTok{(broom)}
\FunctionTok{library}\NormalTok{(dplyr)}
\FunctionTok{library}\NormalTok{(knitr)}
\FunctionTok{library}\NormalTok{(kableExtra)}
\end{Highlighting}
\end{Shaded}

\begin{verbatim}
## 
## Attaching package: 'kableExtra'
\end{verbatim}

\begin{verbatim}
## The following object is masked from 'package:dplyr':
## 
##     group_rows
\end{verbatim}

\begin{Shaded}
\begin{Highlighting}[]
\CommentTok{\# Create tidy table with confidence intervals}
\NormalTok{final\_table\_hx }\OtherTok{\textless{}{-}} \FunctionTok{tidy}\NormalTok{(model\_hx\_final, }\AttributeTok{conf.int =} \ConstantTok{TRUE}\NormalTok{) }\SpecialCharTok{\%\textgreater{}\%}
  \FunctionTok{mutate}\NormalTok{(}
    \AttributeTok{Estimate =} \FunctionTok{round}\NormalTok{(estimate, }\DecValTok{3}\NormalTok{),}
    \AttributeTok{Std.Error =} \FunctionTok{round}\NormalTok{(std.error, }\DecValTok{3}\NormalTok{),}
    \AttributeTok{t\_value =} \FunctionTok{round}\NormalTok{(statistic, }\DecValTok{3}\NormalTok{),}
    \AttributeTok{p\_value =} \FunctionTok{case\_when}\NormalTok{(}
\NormalTok{      p.value }\SpecialCharTok{\textless{}} \FloatTok{0.001} \SpecialCharTok{\textasciitilde{}} \StringTok{"\textless{}0.001"}\NormalTok{,}
      \ConstantTok{TRUE} \SpecialCharTok{\textasciitilde{}} \FunctionTok{sprintf}\NormalTok{(}\StringTok{"\%.3f"}\NormalTok{, p.value)}
\NormalTok{    ),}
    \StringTok{\textasciigrave{}}\AttributeTok{95\% CI}\StringTok{\textasciigrave{}} \OtherTok{=} \FunctionTok{paste0}\NormalTok{(}\StringTok{"["}\NormalTok{, }\FunctionTok{round}\NormalTok{(conf.low, }\DecValTok{3}\NormalTok{), }\StringTok{", "}\NormalTok{, }\FunctionTok{round}\NormalTok{(conf.high, }\DecValTok{3}\NormalTok{), }\StringTok{"]"}\NormalTok{)}
\NormalTok{  ) }\SpecialCharTok{\%\textgreater{}\%}
  \FunctionTok{select}\NormalTok{(term, Estimate, Std.Error, t\_value, p\_value, }\StringTok{\textasciigrave{}}\AttributeTok{95\% CI}\StringTok{\textasciigrave{}}\NormalTok{)}

\CommentTok{\# Clean variable labels}
\NormalTok{final\_table\_hx}\SpecialCharTok{$}\NormalTok{term }\OtherTok{\textless{}{-}} \FunctionTok{recode}\NormalTok{(final\_table\_hx}\SpecialCharTok{$}\NormalTok{term,}
    \StringTok{"(Intercept)"} \OtherTok{=} \StringTok{"Intercept"}\NormalTok{,}
    \StringTok{"ETHNICITYNot Hispanic/Latino"} \OtherTok{=} \StringTok{"Ethnicity: Not Hispanic/Latino"}\NormalTok{,}
    \StringTok{"CESDR\_TOTAL\_SUM"} \OtherTok{=} \StringTok{"Depression (CESD{-}R)"}\NormalTok{,}
    \StringTok{"ACES\_\_\_2"} \OtherTok{=} \StringTok{"ACE 2"}\NormalTok{,}
    \StringTok{"ACES\_\_\_3"} \OtherTok{=} \StringTok{"ACE 3"}\NormalTok{,}
    \StringTok{"ACES\_\_\_7"} \OtherTok{=} \StringTok{"ACE 7"}\NormalTok{,}
    \StringTok{"ACES\_\_\_8"} \OtherTok{=} \StringTok{"ACE 8"}\NormalTok{,}
    \StringTok{"ACES\_\_\_1"} \OtherTok{=} \StringTok{"ACE 1"}\NormalTok{,}
    \StringTok{"ACES\_\_\_10"} \OtherTok{=} \StringTok{"ACE 10"}
\NormalTok{)}

\CommentTok{\# Output decorated table}
\FunctionTok{library}\NormalTok{(flextable)}
\end{Highlighting}
\end{Shaded}

\begin{verbatim}
## 
## Attaching package: 'flextable'
\end{verbatim}

\begin{verbatim}
## The following objects are masked from 'package:kableExtra':
## 
##     as_image, footnote
\end{verbatim}

\begin{Shaded}
\begin{Highlighting}[]
\NormalTok{ft }\OtherTok{\textless{}{-}} \FunctionTok{flextable}\NormalTok{(final\_table\_hx)}
\NormalTok{ft }\OtherTok{\textless{}{-}} \FunctionTok{autofit}\NormalTok{(ft)}
\NormalTok{ft }\OtherTok{\textless{}{-}} \FunctionTok{add\_header\_lines}\NormalTok{(ft, }\StringTok{"Table 2D. Final Multivariable Regression Model Predicting History of Suicide"}\NormalTok{)}

\NormalTok{ft }\OtherTok{\textless{}{-}} \FunctionTok{footnote}\NormalTok{(}
\NormalTok{  ft,}
  \AttributeTok{i =} \DecValTok{1}\NormalTok{, }\AttributeTok{j =} \DecValTok{1}\NormalTok{,}
  \AttributeTok{value =} \FunctionTok{as\_paragraph}\NormalTok{(}
    \StringTok{"Note: Coefficients represent regression estimates predicting HX\_SUICIDE. Significance indicated by p{-}values."}
\NormalTok{  ),}
  \AttributeTok{ref\_symbols =} \StringTok{"a"}
\NormalTok{)}

\NormalTok{ft}
\end{Highlighting}
\end{Shaded}

\global\setlength{\Oldarrayrulewidth}{\arrayrulewidth}

\global\setlength{\Oldtabcolsep}{\tabcolsep}

\setlength{\tabcolsep}{2pt}

\renewcommand*{\arraystretch}{1.5}



\providecommand{\ascline}[3]{\noalign{\global\arrayrulewidth #1}\arrayrulecolor[HTML]{#2}\cline{#3}}

\begin{longtable}[c]{|p{2.27in}|p{0.88in}|p{0.90in}|p{0.78in}|p{0.82in}|p{1.32in}}



\ascline{1.5pt}{666666}{1-6}

\multicolumn{6}{>{\raggedright}m{\dimexpr 6.97in+10\tabcolsep}}{\textcolor[HTML]{000000}{\fontsize{11}{11}\selectfont{\global\setmainfont{Arial}{Table\ 2D.\ Final\ Multivariable\ Regression\ Model\ Predicting\ History\ of\ Suicide}}}} \\

\ascline{1.5pt}{666666}{1-6}



\multicolumn{1}{>{\raggedright}m{\dimexpr 2.27in+0\tabcolsep}}{\textcolor[HTML]{000000}{\fontsize{11}{11}\selectfont{\global\setmainfont{Arial}{term}}}} & \multicolumn{1}{>{\raggedleft}m{\dimexpr 0.88in+0\tabcolsep}}{\textcolor[HTML]{000000}{\fontsize{11}{11}\selectfont{\global\setmainfont{Arial}{Estimate}}}} & \multicolumn{1}{>{\raggedleft}m{\dimexpr 0.9in+0\tabcolsep}}{\textcolor[HTML]{000000}{\fontsize{11}{11}\selectfont{\global\setmainfont{Arial}{Std.Error}}}} & \multicolumn{1}{>{\raggedleft}m{\dimexpr 0.78in+0\tabcolsep}}{\textcolor[HTML]{000000}{\fontsize{11}{11}\selectfont{\global\setmainfont{Arial}{t\_value}}}} & \multicolumn{1}{>{\raggedright}m{\dimexpr 0.82in+0\tabcolsep}}{\textcolor[HTML]{000000}{\fontsize{11}{11}\selectfont{\global\setmainfont{Arial}{p\_value}}}} & \multicolumn{1}{>{\raggedright}m{\dimexpr 1.32in+0\tabcolsep}}{\textcolor[HTML]{000000}{\fontsize{11}{11}\selectfont{\global\setmainfont{Arial}{95\%\ CI}}}} \\

\ascline{1.5pt}{666666}{1-6}\endfirsthead 

\ascline{1.5pt}{666666}{1-6}

\multicolumn{6}{>{\raggedright}m{\dimexpr 6.97in+10\tabcolsep}}{\textcolor[HTML]{000000}{\fontsize{11}{11}\selectfont{\global\setmainfont{Arial}{Table\ 2D.\ Final\ Multivariable\ Regression\ Model\ Predicting\ History\ of\ Suicide}}}} \\

\ascline{1.5pt}{666666}{1-6}



\multicolumn{1}{>{\raggedright}m{\dimexpr 2.27in+0\tabcolsep}}{\textcolor[HTML]{000000}{\fontsize{11}{11}\selectfont{\global\setmainfont{Arial}{term}}}} & \multicolumn{1}{>{\raggedleft}m{\dimexpr 0.88in+0\tabcolsep}}{\textcolor[HTML]{000000}{\fontsize{11}{11}\selectfont{\global\setmainfont{Arial}{Estimate}}}} & \multicolumn{1}{>{\raggedleft}m{\dimexpr 0.9in+0\tabcolsep}}{\textcolor[HTML]{000000}{\fontsize{11}{11}\selectfont{\global\setmainfont{Arial}{Std.Error}}}} & \multicolumn{1}{>{\raggedleft}m{\dimexpr 0.78in+0\tabcolsep}}{\textcolor[HTML]{000000}{\fontsize{11}{11}\selectfont{\global\setmainfont{Arial}{t\_value}}}} & \multicolumn{1}{>{\raggedright}m{\dimexpr 0.82in+0\tabcolsep}}{\textcolor[HTML]{000000}{\fontsize{11}{11}\selectfont{\global\setmainfont{Arial}{p\_value}}}} & \multicolumn{1}{>{\raggedright}m{\dimexpr 1.32in+0\tabcolsep}}{\textcolor[HTML]{000000}{\fontsize{11}{11}\selectfont{\global\setmainfont{Arial}{95\%\ CI}}}} \\

\ascline{1.5pt}{666666}{1-6}\endhead



\multicolumn{6}{>{\raggedright}m{\dimexpr 6.97in+10\tabcolsep}}{\textcolor[HTML]{000000}{\fontsize{11}{11}\selectfont{\global\setmainfont{Arial}{\textsuperscript{a}}}}\textcolor[HTML]{000000}{\fontsize{11}{11}\selectfont{\global\setmainfont{Arial}{Note:\ Coefficients\ represent\ regression\ estimates\ predicting\ HX\_SUICIDE.\ Significance\ indicated\ by\ p-values.}}}} \\

\endlastfoot



\multicolumn{1}{>{\raggedright}m{\dimexpr 2.27in+0\tabcolsep}}{\textcolor[HTML]{000000}{\fontsize{11}{11}\selectfont{\global\setmainfont{Arial}{Intercept}}}\textcolor[HTML]{000000}{\fontsize{11}{11}\selectfont{\global\setmainfont{Arial}{\textsuperscript{a}}}}} & \multicolumn{1}{>{\raggedleft}m{\dimexpr 0.88in+0\tabcolsep}}{\textcolor[HTML]{000000}{\fontsize{11}{11}\selectfont{\global\setmainfont{Arial}{0.033}}}} & \multicolumn{1}{>{\raggedleft}m{\dimexpr 0.9in+0\tabcolsep}}{\textcolor[HTML]{000000}{\fontsize{11}{11}\selectfont{\global\setmainfont{Arial}{0.036}}}} & \multicolumn{1}{>{\raggedleft}m{\dimexpr 0.78in+0\tabcolsep}}{\textcolor[HTML]{000000}{\fontsize{11}{11}\selectfont{\global\setmainfont{Arial}{0.939}}}} & \multicolumn{1}{>{\raggedright}m{\dimexpr 0.82in+0\tabcolsep}}{\textcolor[HTML]{000000}{\fontsize{11}{11}\selectfont{\global\setmainfont{Arial}{0.348}}}} & \multicolumn{1}{>{\raggedright}m{\dimexpr 1.32in+0\tabcolsep}}{\textcolor[HTML]{000000}{\fontsize{11}{11}\selectfont{\global\setmainfont{Arial}{[-0.036,\ 0.103]}}}} \\





\multicolumn{1}{>{\raggedright}m{\dimexpr 2.27in+0\tabcolsep}}{\textcolor[HTML]{000000}{\fontsize{11}{11}\selectfont{\global\setmainfont{Arial}{Ethnicity:\ Not\ Hispanic/Latino}}}} & \multicolumn{1}{>{\raggedleft}m{\dimexpr 0.88in+0\tabcolsep}}{\textcolor[HTML]{000000}{\fontsize{11}{11}\selectfont{\global\setmainfont{Arial}{-0.070}}}} & \multicolumn{1}{>{\raggedleft}m{\dimexpr 0.9in+0\tabcolsep}}{\textcolor[HTML]{000000}{\fontsize{11}{11}\selectfont{\global\setmainfont{Arial}{0.034}}}} & \multicolumn{1}{>{\raggedleft}m{\dimexpr 0.78in+0\tabcolsep}}{\textcolor[HTML]{000000}{\fontsize{11}{11}\selectfont{\global\setmainfont{Arial}{-2.038}}}} & \multicolumn{1}{>{\raggedright}m{\dimexpr 0.82in+0\tabcolsep}}{\textcolor[HTML]{000000}{\fontsize{11}{11}\selectfont{\global\setmainfont{Arial}{0.042}}}} & \multicolumn{1}{>{\raggedright}m{\dimexpr 1.32in+0\tabcolsep}}{\textcolor[HTML]{000000}{\fontsize{11}{11}\selectfont{\global\setmainfont{Arial}{[-0.138,\ -0.003]}}}} \\





\multicolumn{1}{>{\raggedright}m{\dimexpr 2.27in+0\tabcolsep}}{\textcolor[HTML]{000000}{\fontsize{11}{11}\selectfont{\global\setmainfont{Arial}{Depression\ (CESD-R)}}}} & \multicolumn{1}{>{\raggedleft}m{\dimexpr 0.88in+0\tabcolsep}}{\textcolor[HTML]{000000}{\fontsize{11}{11}\selectfont{\global\setmainfont{Arial}{0.004}}}} & \multicolumn{1}{>{\raggedleft}m{\dimexpr 0.9in+0\tabcolsep}}{\textcolor[HTML]{000000}{\fontsize{11}{11}\selectfont{\global\setmainfont{Arial}{0.001}}}} & \multicolumn{1}{>{\raggedleft}m{\dimexpr 0.78in+0\tabcolsep}}{\textcolor[HTML]{000000}{\fontsize{11}{11}\selectfont{\global\setmainfont{Arial}{4.644}}}} & \multicolumn{1}{>{\raggedright}m{\dimexpr 0.82in+0\tabcolsep}}{\textcolor[HTML]{000000}{\fontsize{11}{11}\selectfont{\global\setmainfont{Arial}{<0.001}}}} & \multicolumn{1}{>{\raggedright}m{\dimexpr 1.32in+0\tabcolsep}}{\textcolor[HTML]{000000}{\fontsize{11}{11}\selectfont{\global\setmainfont{Arial}{[0.002,\ 0.006]}}}} \\





\multicolumn{1}{>{\raggedright}m{\dimexpr 2.27in+0\tabcolsep}}{\textcolor[HTML]{000000}{\fontsize{11}{11}\selectfont{\global\setmainfont{Arial}{ACE\ 2}}}} & \multicolumn{1}{>{\raggedleft}m{\dimexpr 0.88in+0\tabcolsep}}{\textcolor[HTML]{000000}{\fontsize{11}{11}\selectfont{\global\setmainfont{Arial}{0.166}}}} & \multicolumn{1}{>{\raggedleft}m{\dimexpr 0.9in+0\tabcolsep}}{\textcolor[HTML]{000000}{\fontsize{11}{11}\selectfont{\global\setmainfont{Arial}{0.033}}}} & \multicolumn{1}{>{\raggedleft}m{\dimexpr 0.78in+0\tabcolsep}}{\textcolor[HTML]{000000}{\fontsize{11}{11}\selectfont{\global\setmainfont{Arial}{5.058}}}} & \multicolumn{1}{>{\raggedright}m{\dimexpr 0.82in+0\tabcolsep}}{\textcolor[HTML]{000000}{\fontsize{11}{11}\selectfont{\global\setmainfont{Arial}{<0.001}}}} & \multicolumn{1}{>{\raggedright}m{\dimexpr 1.32in+0\tabcolsep}}{\textcolor[HTML]{000000}{\fontsize{11}{11}\selectfont{\global\setmainfont{Arial}{[0.101,\ 0.23]}}}} \\





\multicolumn{1}{>{\raggedright}m{\dimexpr 2.27in+0\tabcolsep}}{\textcolor[HTML]{000000}{\fontsize{11}{11}\selectfont{\global\setmainfont{Arial}{ACE\ 3}}}} & \multicolumn{1}{>{\raggedleft}m{\dimexpr 0.88in+0\tabcolsep}}{\textcolor[HTML]{000000}{\fontsize{11}{11}\selectfont{\global\setmainfont{Arial}{0.045}}}} & \multicolumn{1}{>{\raggedleft}m{\dimexpr 0.9in+0\tabcolsep}}{\textcolor[HTML]{000000}{\fontsize{11}{11}\selectfont{\global\setmainfont{Arial}{0.031}}}} & \multicolumn{1}{>{\raggedleft}m{\dimexpr 0.78in+0\tabcolsep}}{\textcolor[HTML]{000000}{\fontsize{11}{11}\selectfont{\global\setmainfont{Arial}{1.445}}}} & \multicolumn{1}{>{\raggedright}m{\dimexpr 0.82in+0\tabcolsep}}{\textcolor[HTML]{000000}{\fontsize{11}{11}\selectfont{\global\setmainfont{Arial}{0.149}}}} & \multicolumn{1}{>{\raggedright}m{\dimexpr 1.32in+0\tabcolsep}}{\textcolor[HTML]{000000}{\fontsize{11}{11}\selectfont{\global\setmainfont{Arial}{[-0.016,\ 0.107]}}}} \\





\multicolumn{1}{>{\raggedright}m{\dimexpr 2.27in+0\tabcolsep}}{\textcolor[HTML]{000000}{\fontsize{11}{11}\selectfont{\global\setmainfont{Arial}{ACE\ 7}}}} & \multicolumn{1}{>{\raggedleft}m{\dimexpr 0.88in+0\tabcolsep}}{\textcolor[HTML]{000000}{\fontsize{11}{11}\selectfont{\global\setmainfont{Arial}{-0.076}}}} & \multicolumn{1}{>{\raggedleft}m{\dimexpr 0.9in+0\tabcolsep}}{\textcolor[HTML]{000000}{\fontsize{11}{11}\selectfont{\global\setmainfont{Arial}{0.037}}}} & \multicolumn{1}{>{\raggedleft}m{\dimexpr 0.78in+0\tabcolsep}}{\textcolor[HTML]{000000}{\fontsize{11}{11}\selectfont{\global\setmainfont{Arial}{-2.042}}}} & \multicolumn{1}{>{\raggedright}m{\dimexpr 0.82in+0\tabcolsep}}{\textcolor[HTML]{000000}{\fontsize{11}{11}\selectfont{\global\setmainfont{Arial}{0.042}}}} & \multicolumn{1}{>{\raggedright}m{\dimexpr 1.32in+0\tabcolsep}}{\textcolor[HTML]{000000}{\fontsize{11}{11}\selectfont{\global\setmainfont{Arial}{[-0.149,\ -0.003]}}}} \\





\multicolumn{1}{>{\raggedright}m{\dimexpr 2.27in+0\tabcolsep}}{\textcolor[HTML]{000000}{\fontsize{11}{11}\selectfont{\global\setmainfont{Arial}{ACE\ 8}}}} & \multicolumn{1}{>{\raggedleft}m{\dimexpr 0.88in+0\tabcolsep}}{\textcolor[HTML]{000000}{\fontsize{11}{11}\selectfont{\global\setmainfont{Arial}{0.090}}}} & \multicolumn{1}{>{\raggedleft}m{\dimexpr 0.9in+0\tabcolsep}}{\textcolor[HTML]{000000}{\fontsize{11}{11}\selectfont{\global\setmainfont{Arial}{0.031}}}} & \multicolumn{1}{>{\raggedleft}m{\dimexpr 0.78in+0\tabcolsep}}{\textcolor[HTML]{000000}{\fontsize{11}{11}\selectfont{\global\setmainfont{Arial}{2.883}}}} & \multicolumn{1}{>{\raggedright}m{\dimexpr 0.82in+0\tabcolsep}}{\textcolor[HTML]{000000}{\fontsize{11}{11}\selectfont{\global\setmainfont{Arial}{0.004}}}} & \multicolumn{1}{>{\raggedright}m{\dimexpr 1.32in+0\tabcolsep}}{\textcolor[HTML]{000000}{\fontsize{11}{11}\selectfont{\global\setmainfont{Arial}{[0.029,\ 0.151]}}}} \\





\multicolumn{1}{>{\raggedright}m{\dimexpr 2.27in+0\tabcolsep}}{\textcolor[HTML]{000000}{\fontsize{11}{11}\selectfont{\global\setmainfont{Arial}{ACE\ 1}}}} & \multicolumn{1}{>{\raggedleft}m{\dimexpr 0.88in+0\tabcolsep}}{\textcolor[HTML]{000000}{\fontsize{11}{11}\selectfont{\global\setmainfont{Arial}{-0.025}}}} & \multicolumn{1}{>{\raggedleft}m{\dimexpr 0.9in+0\tabcolsep}}{\textcolor[HTML]{000000}{\fontsize{11}{11}\selectfont{\global\setmainfont{Arial}{0.040}}}} & \multicolumn{1}{>{\raggedleft}m{\dimexpr 0.78in+0\tabcolsep}}{\textcolor[HTML]{000000}{\fontsize{11}{11}\selectfont{\global\setmainfont{Arial}{-0.627}}}} & \multicolumn{1}{>{\raggedright}m{\dimexpr 0.82in+0\tabcolsep}}{\textcolor[HTML]{000000}{\fontsize{11}{11}\selectfont{\global\setmainfont{Arial}{0.531}}}} & \multicolumn{1}{>{\raggedright}m{\dimexpr 1.32in+0\tabcolsep}}{\textcolor[HTML]{000000}{\fontsize{11}{11}\selectfont{\global\setmainfont{Arial}{[-0.105,\ 0.054]}}}} \\





\multicolumn{1}{>{\raggedright}m{\dimexpr 2.27in+0\tabcolsep}}{\textcolor[HTML]{000000}{\fontsize{11}{11}\selectfont{\global\setmainfont{Arial}{ACE\ 10}}}} & \multicolumn{1}{>{\raggedleft}m{\dimexpr 0.88in+0\tabcolsep}}{\textcolor[HTML]{000000}{\fontsize{11}{11}\selectfont{\global\setmainfont{Arial}{0.125}}}} & \multicolumn{1}{>{\raggedleft}m{\dimexpr 0.9in+0\tabcolsep}}{\textcolor[HTML]{000000}{\fontsize{11}{11}\selectfont{\global\setmainfont{Arial}{0.057}}}} & \multicolumn{1}{>{\raggedleft}m{\dimexpr 0.78in+0\tabcolsep}}{\textcolor[HTML]{000000}{\fontsize{11}{11}\selectfont{\global\setmainfont{Arial}{2.220}}}} & \multicolumn{1}{>{\raggedright}m{\dimexpr 0.82in+0\tabcolsep}}{\textcolor[HTML]{000000}{\fontsize{11}{11}\selectfont{\global\setmainfont{Arial}{0.027}}}} & \multicolumn{1}{>{\raggedright}m{\dimexpr 1.32in+0\tabcolsep}}{\textcolor[HTML]{000000}{\fontsize{11}{11}\selectfont{\global\setmainfont{Arial}{[0.014,\ 0.237]}}}} \\

\ascline{1.5pt}{666666}{1-6}



\end{longtable}



\arrayrulecolor[HTML]{000000}

\global\setlength{\arrayrulewidth}{\Oldarrayrulewidth}

\global\setlength{\tabcolsep}{\Oldtabcolsep}

\renewcommand*{\arraystretch}{1}

The final multivariable model, which includes all the factors chosen at
α = 0.20, explains around 17\% of the differences in suicide history,
which is better than any of the individual models. Depression continues
to be a robust and substantial predictor, underscoring the importance of
psychological distress in comprehending suicide risk. Certain ACE
components, specifically ACEs 2, 7, 8, and 10, maintained their
significance, indicating that particular childhood traumas persist in
affecting suicide history, even after controlling for depression and
demographic variables. Ethnicity continued to be a prominent demographic
variable. These results indicate that a history of suicide is affected
by current mental health and past life challenges.

\begin{Shaded}
\begin{Highlighting}[]
\CommentTok{\# ============================}
\CommentTok{\# PART E}
\CommentTok{\# ============================}
\CommentTok{\# Diagnostics}
\FunctionTok{par}\NormalTok{(}\AttributeTok{mfrow=}\FunctionTok{c}\NormalTok{(}\DecValTok{2}\NormalTok{,}\DecValTok{2}\NormalTok{))}
\FunctionTok{plot}\NormalTok{(model\_hx\_final)}
\end{Highlighting}
\end{Shaded}

\pandocbounded{\includegraphics[keepaspectratio]{Activity_4_files/figure-latex/unnamed-chunk-12-1.pdf}}

\begin{Shaded}
\begin{Highlighting}[]
\FunctionTok{par}\NormalTok{(}\AttributeTok{mfrow=}\FunctionTok{c}\NormalTok{(}\DecValTok{1}\NormalTok{,}\DecValTok{1}\NormalTok{))}

\CommentTok{\# Normality test}
\FunctionTok{shapiro.test}\NormalTok{(}\FunctionTok{resid}\NormalTok{(model\_hx\_final))}
\end{Highlighting}
\end{Shaded}

\begin{verbatim}
## 
##  Shapiro-Wilk normality test
## 
## data:  resid(model_hx_final)
## W = 0.71189, p-value < 2.2e-16
\end{verbatim}

The residual diagnostics indicate that the model does not completely
satisfy the assumption of normality, as validated by the Shapiro-Wilk
test (p \textless{} .00000000000000022). This is to be expected because
HX\_SUICIDE is almost binary (0/1-like), which means that complete
normal residuals are not statistically possible. Even yet, the model
still gives useful information on what causes people to commit suicide.
The residual plot does not show any strong signs of heteroscedasticity
or extreme outliers, which means that the model is quite stable.
Overall, the model is still useful for finding crucial correlates, even
though the normalcy assumption isn't ideal.

\section{QUESTION 3.}\label{question-3.}

\subsection{From Model 1d (SABCS Final
Model)}\label{from-model-1d-sabcs-final-model}

✔ Significant Predictor: CESDR\_TOTAL\_SUM

Beta: 0.18484

Interpretation: Each additional point in depression score is associated
with a 0.18-point increase in suicidal risk. This means depressive
symptoms are a strong, meaningful predictor of higher SABCS scores.

✔ Non-significant Predictor: ACES\_1

Beta: 0.11834

Interpretation: Although ACES\_1 has a small positive coefficient, it is
not statistically significant, meaning it does not meaningfully predict
suicidal risk. Any observed relationship may simply be due to chance.

\subsection{From Model 2d (HX\_SUICIDE Final
Model)}\label{from-model-2d-hx_suicide-final-model}

✔ Significant Predictor: CESDR\_TOTAL\_SUM

Beta: 0.004033

Interpretation: Each 1-point increase in depression score is associated
with a small increase in suicide history score, meaning individuals with
more depressive symptoms are more likely to report a prior suicide
attempt or ideation.

✔ Non-significant Predictor: ACES\_1

Beta: --0.02537

Interpretation: ACES\_1 is not statistically significant, indicating
that this type of childhood adversity does not meaningfully predict past
suicidal behavior when other variables are considered.

\section{QUESTION 4.}\label{question-4.}

\subsection{1. Study Design}\label{study-design}

The structure of this dataset aligns most closely with a cross-sectional
study, since all variables---life experiences, mental health measures,
and suicide-related outcomes---were collected at a single point in time.
It feels like a snapshot of each participant's life at that moment,
capturing their history and current emotional state simultaneously.
Because nothing is measured across time, it does not function like a
cohort or case-control design.

\subsection{2. Limitation (1--2 sentences, humanized
English)}\label{limitation-12-sentences-humanized-english}

A major limitation of this ``single-moment'' approach is that it cannot
reveal which experiences came first, making it hard to understand the
true sequence of events. Without a sense of timing, any relationship we
see may reflect correlation rather than a meaningful causal story.

\subsection{3. How to Redesign the Study (3--4 sentences, humanized
English)}\label{how-to-redesign-the-study-34-sentences-humanized-english}

To address this issue, the study could be redesigned as a longitudinal
cohort, where participants are followed over time and their emotional
and environmental changes are recorded as they unfold. Tracking people
across meaningful life stages would help establish whether depression,
stress, or adversity occur before shifts in suicide risk. This design
respects the fact that human experiences evolve rather than appear all
at once. By observing these changes directly, we gain a clearer and more
compassionate understanding of how struggles develop and how risks
emerge over time.

\end{document}
